\section{Введение}
Данное приложение ищет вирусы на ПК. Приложение позволяет пользователю проверить конкретный файл либо директорию целиком. Для сканирования используется база вирусов, предварительно скачанная с сети Интернет. Журнал с результатом сканирования отображается в окне приложения.
\section{Отказ от ответственности}
Данная спецификация не является итоговой (законченной). Если у Вас есть вопросы, предложения или Вы нашли ошибку, пожалуйста, свяжитесь со мной по e-mail: alexey@bogdanenko.com
\section{Автор}
Алексей Богданенко, скрам-мастер
\section{Ключевые требования}
\subsection{Функциональные требования и архитектура}
Архитектура приложения, а также предназначение основных классов приведены в документации в папке help (основной файл — index.html). Функциональные требования приведены в Приложении 1.
\subsection{Пользовательские требования}
\begin{enumerate}
\item При запуске Программа восстанавливает своё состояние на момент последнего запуска: выбранную пользователем директорию (файл), настройки сканирования, журнал.
\item В Программе предусмотрен графический диалог выбора директории (файла) для сканирования.
\item В окне Программы отображается журнал, информирующий Пользователя о работе Программы. При желании Пользователь может очистить журнал.
\item Программа находит вирусы и поступает с ними так, как указал Пользователь при запуске сканирования: удаляет либо игнорирует.
\end{enumerate}
\subsection{Нефункциональные требования}
\begin{itemize}
\item Программа не зависает во время выполнения длительных операций
\item Программа занимает не более 512МБ оперативной памяти
\item Для работы Программы не требуются привеллегии суперпользователя, в случае, если сканируемые файлы доступны для чтения и записи Пользователю.
\end{itemize}
\section{Команда разработки}
Состав команды приведён в таблице \ref{table:team}.
\begin{table}[h]
\centering
\begin{tabular}{|p{25mm}|p{35mm}|l|p{25mm}|}
\hline
ФИО & Роли & Эл. почта & Учётная запись BitBucket \\
\hline
Богданенко А.~О. & Скрам-мастер, техн. писатель, GUI-разработчик & alexey@bogdanenko.com & abogdanenko \\
\hline
Хамитов К.~Г. & Гл. разработчик & berserq0123@gmail.com & lberserq \\
\hline
\end{tabular}
\caption{Состав команды}
\label{table:team}
\end{table}
Распределение компонентов между разработчиками приведено в Приложении 1. Подробное распределение обязанностей приведено в Приложении 2.
\section{Типовые сценарии (пользовательские истории)}
\begin{enumerate}
\item Пользователь выбирает директорию (файл) для сканирования. Пользователь выбирает, что делать с заражёнными файлами. Пользователь запускает сканирование. Программа сканирует файлы в выбранной директории (выбранный файл), информирует Пользователя о состоянии работы в журнале. Журнал отображается во время работы Программы в окне Программы.
\item Пользователь закрыл Программу. Пользователь открыл Программу. Программа сохранила выбранную пользователем директорию (файл), настройки сканирования, журнал.
\item Пользователь пожелал очистить журнал, и в окне Программы нашлась для этого кнопка.
\item Пользователь запустил сканирование директории с 8-гигабайтными HD-видео файлами. Программа не зависла и не завершилась аварийно.
\item Пользователь запустил сканирование системной директории, на которую у него нет прав. Программа проинформировала Пользователя на человеческом языке о проблеме при сканировании.
\end{enumerate}
\section{Экранные формы}
\section{Антитребования}
\section{Открытые задачи}
\section{Идеи}
\section{Журнал}
\section{Приложения}
Приложение 1. Компоненты и распределение базовой функциональности сканера за разработчиками.

Распределение компонентов между разработчиками приведено в таблице \ref{table:responsibilities}.
\begin{table}[h]
\centering
\begin{tabular}{|l|p{3cm}|p{3cm}|}
\hline
Разработчик & Входит в группу / является & Компоненты \\
\hline
Богданенко А. О. & Скрам-мастер & Графический интерфейс, Настройки, Прочее \\
\hline
Хамитов К. Г. & Скрам-команда & Обработчик вредоносных программ, Файловый браузер, Сигнатурный анализатор, Сканер \\
\hline
\end{tabular}
\caption{Распределение компонентов между разработчиками}
\label{table:responsibilities}
\end{table}
