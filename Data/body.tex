\section{Введение}
Данное приложение ищет вирусы на ПК. Приложение позволяет пользователю проверить конкретный файл либо директорию целиком. Для сканирования используется база вирусов, предварительно скачанная с сети Интернет. Журнал с результатом сканирования отображается в окне приложения.
\section{Отказ от ответственности}
Данная спецификация не является итоговой (законченной). Если у Вас есть вопросы, предложения или Вы нашли ошибку, пожалуйста, свяжитесь со мной по e-mail: alexey@bogdanenko.com
\section{Автор}
Алексей Богданенко, скрам-мастер
\section{Ключевые требования}
\subsection{Функциональные требования и архитектура}
\subsection{Пользовательские требования}
\subsection{Нефункциональные требования}
\section{Команда разработки}
\section{Типовые сценарии (пользовательские истории)}
\begin{enumerate}
\item Пользователь выбирает директорию (файл) для сканирования. Пользователь выбирает, что делать с заражёнными файлами. Пользователь запускает сканирование. Программа сканирует файлы в выбранной директории (выбранный файл), информирует Пользователя о состоянии работы в журнале. Журнал отображается во время работы Программы в окне Программы.
\item Пользователь закрыл Программу. Пользователь открыл Программу. Программа сохранила выбранную пользователем директорию (файл), настройки сканирования, журнал.
\item Пользователь пожелал очистить журнал, и в окне Программы нашлась для этого кнопка.
\item Пользователь запустил сканирование директории с 8-гигабайтными HD-видео файлами. Программа не зависла и не завершилась аварийно.
\item Пользователь запустил сканирование системной директории, на которую у него нет прав. Программа проинформировала Пользователя на человеческом языке о проблеме при сканировании.
\end{enumerate}
\section{Экранные формы}
\section{Антитребования}
\section{Открытые задачи}
\section{Идеи}
\section{Журнал}
\section{Приложения}
